%% Copyright (c) 2002, 2010 Sam Williams
%% Copyright (c) 2010 Richard M. Stallman
%% Permission is granted to copy, distribute and/or modify this
%% document under the terms of the GNU Free Documentation License,
%% Version 1.3 or any later version published by the Free Software
%% Foundation; with no Invariant Sections, no Front-Cover Texts, and
%% no Back-Cover Texts. A copy of the license is included in the
%% file called ``gfdl.tex''.


\chapter{Continuing the Fight}

For Richard Stallman, time may not heal all wounds, but it does provide a convenient ally.

Four years after ``The Cathedral and the Bazaar,'' Stallman still chafes over the Raymond critique. He also grumbles over Linus Torvalds' elevation to the role of world's most famous hacker. He recalls a popular T-shirt that began showing at Linux tradeshows around 1999. Designed to mimic the original promotional poster for Star Wars, the shirt depicted Torvalds brandishing a lightsaber like Luke Skywalker, while Stallman's face rides atop R2D2. The shirt still grates on Stallman's nerves not only because it depicts him as Torvalds' sidekick, but also because it elevates Torvalds to the leadership role in the free software community, a role even Torvalds himself is loath to accept. ``It's ironic,'' says Stallman mournfully. ``Picking up that sword is exactly what Linus refuses to do. He gets everybody focusing on him as the symbol of the movement, and then he won't fight. What good is it?''

Then again, it is that same unwillingness to ``pick up the sword,'' on Torvalds' part, that has left the door open for Stallman to bolster his reputation as the hacker community's ethical arbiter. Despite his grievances, Stallman has to admit that the last few years have been quite good, both to himself and to his organization. Relegated to the periphery by the ironic success of the GNU/Linux system because users thought of it as ``Linux,'' Stallman has nonetheless successfully recaptured the initiative. His speaking schedule between January 2000 and December 2001 included stops on six continents and visits to countries where the notion of software freedom carries heavy overtones -- China and India, for example.

Outside the bully pulpit, Stallman has taken advantage of the leverage of the GNU General Public License (GPL), of which he remains the steward. During the summer of 2000, while the air was rapidly leaking out of the 1999 Linux IPO bubble, Stallman and the Free Software Foundation scored two major victories. In July, 2000, Trolltech, a Norwegian software company and developer of Qt, a graphical interface library that ran on the GNU/Linux operating system, announced it was licensing its software under the GPL. A few weeks later, Sun Microsystems, a company that, until then, had been warily trying to ride the open source bandwagon without actually contributing its code, finally relented and announced that it, too, was dual licensing its new OpenOffice\endnote{Sun was compelled by a trademark complaint to use the clumsy name ``OpenOffice.org.''} application suite under the Lesser GNU Public License (LGPL) and the Sun Industry Standards Source License (SISSL).

In the case of Trolltech, this victory was the result of a protracted effort by the GNU Project. The nonfreeness of Qt was a serious problem for the free software community because KDE, a free graphical desktop environment that was becoming popular, depended on it. Qt was non-free software but Trolltech had invited free software projects (such as KDE) to use it gratis. Although KDE itself was free software, users that insisted on freedom couldn't run it, since they had to reject Qt. Stallman recognized that many users would want a graphical desktop on GNU/Linux, and most would not value freedom enough to reject the temptation of KDE, with Qt hiding within. The danger was that GNU/Linux would become a motor for the installation of KDE, and therefore also of non-free Qt. This would undermine the freedom which was the purpose of GNU.

To deal with this danger, Stallman recruited people to launch two parallel counterprojects. One was GNOME, the GNU free graphical desktop environment. The other was Harmony, a compatible free replacement for Qt. If GNOME succeeded, KDE would not be necessary; if Harmony succeeded, KDE would not need Qt. Either way, users would be able to have a graphical desktop on GNU/Linux without nonfree Qt.

In 1999, these two efforts were making good progress, and the management of Trolltech were starting to feel the pressure. So Trolltech released Qt under its own free software license, the QPL. The QPL qualified as a free license, but Stallman pointed out the drawback of incompatibility with the GPL: in general, combining GPL-covered code with Qt in one program was impossible without violating one license or the other. Eventually the Trolltech management recognized that the GPL would serve their purposes equally well, and released Qt with dual licensing: the same Qt code, in parallel, was available under the GNU GPL and under the QPL. After three years, this was victory.

Once Qt was free, the motive for developing Harmony (which wasn't complete enough for actual use) had disappeared, and the developers abandoned it. GNOME had acquired substantial momentum, so its development continued, and it remains the main GNU graphical desktop.

Sun desired to play according to the Free Software Foundation's conditions. At the 1999 O'Reilly Open Source Conference, Sun Microsystems cofounder and chief scientist Bill Joy defended his company's ``community source'' license, essentially a watered-down compromise letting users copy and modify Sun-owned software but not sell copies of said software without negotiating a royalty agreement with Sun. (With this restriction, the license did not qualify as free, nor for that matter as open source.) A year after Joy's speech, Sun Microsystems vice president Marco Boerries was appearing on the same stage spelling out the company's new licensing compromise in the case of OpenOffice, an office-application suite designed specifically for the GNU/Linux operating system.

``I can spell it out in three letters,'' said Boerries. ``GPL.''

At the time, Boerries said his company's decision had little to do with Stallman and more to do with the momentum of GPL-protected programs. ``What basically happened was the recognition that different products attracted different communities, and the license you use depends on what type of community you want to attract,'' said Boerries. ``With [OpenOffice], it was clear we had the highest correlation with the GPL community.''\endnote{Marco Boerries, interview with author (July, 2000).}  Alas, this victory was incomplete, since OpenOffice recommends the use of nonfree plug-ins.

Such comments point out the under-recognized strength of the GPL and, indirectly, the political genius of the man who played the largest role in creating it. ``There isn't a lawyer on earth who would have drafted the GPL the way it is,'' says Eben Moglen, Columbia University law professor and Free Software Foundation general counsel. ``But it works. And it works because of Richard's philosophy of design.''

A former professional programmer, Moglen traces his pro bono work with Stallman back to 1990 when Stallman requested Moglen's legal assistance on a private affair. Moglen, then working with encryption expert Phillip Zimmerman during Zimmerman's legal battles with the federal government, says he was honored by the request.\endnote{For more information on Zimmerman's legal travails, read Steven Levy's \textit{Crypto}, p. 287-288. In the original book version of \textit{Free as in Freedom}, I reported that Moglen helped defend Zimmerman against the National Security Agency. According to Levy's account, Zimmerman was investigated by the U.S. Attorney's office and U.S. Customs, not the NSA.}

``I told him I used Emacs every day of my life, and it would take an awful lot of lawyering on my part to pay off the debt.''

Since then, Moglen, perhaps more than any other individual, has had the best chance to observe the crossover of Stallman's hacker philosophies into the legal realm. Moglen says Stallman's approach to legal code and his approach to software code are largely the same. ``I have to say, as a lawyer, the idea that what you should do with a legal document is to take out all the bugs doesn't make much sense,'' Moglen says. ``There is uncertainty in every legal process, and what most lawyers want to do is to capture the benefits of uncertainty for their client. Richard's goal is the complete opposite. His goal is to remove uncertainty, which is inherently impossible. It is inherently impossible to draft one license to control all circumstances in all legal systems all over the world. But if you were to go at it, you would have to go at it his way. And the resulting elegance, the resulting simplicity in design almost achieves what it has to achieve. And from there a little lawyering will carry you quite far.''

As the person charged with pushing the Stallman agenda, Moglen understands the frustration of would-be allies. ``Richard is a man who does not want to compromise over matters that he thinks of as fundamental,'' Moglen says, ``and he does not take easily the twisting of words or even just the seeking of artful ambiguity, which human society often requires from a lot of people.''

In addition to helping the Free Software Foundation, Moglen has provided legal aid to other copyright defendants, such as Dmitry Sklyarov, and distributors of the DVD decryption program deCSS.

Sklyarov had written and released a program to break digital copy-protection on Adobe e-Books, in Russia where there was no law against it, as an employee of a Russian company.  He was then arrested while visiting the US to give a scientific paper about his work. Stallman eagerly participated in protests condemning Adobe for having Sklyarov arrested, and the Free Software Foundation denounced the Digital Millennium Copyright Act as ``censorship of software,'' but it could not intervene in favor of Sklyarov's program because that was nonfree.  Thus, Moglen worked for Sklyarov's defense through the Electronic Frontier Foundation.  The FSF avoided involvement in the distribution of deCSS, since that was illegal, but Stallman condemned the U.S. government for prohibiting deCSS, and Moglen worked as direct counsel for the defendants.

Following the FSF's decision not to involve itself in those cases, Moglen has learned to appreciate the value of Stallman's stubbornness. ``There have been times over the years where I've gone to Richard and said, `We have to do this. We have to do that. Here's the strategic situation. Here's the next move. Here's what he have to do.' And Richard's response has always been, `We don't have to do anything.' Just wait. What needs doing will get done.''

``And you know what?'' Moglen adds. ``Generally, he's been right.''

Such comments disavow Stallman's own self-assessment: ``I'm not good at playing games,'' Stallman says, addressing the many unseen critics who see him as a shrewd strategist. ``I'm not good at looking ahead and anticipating what somebody else might do. My approach has always been to focus on the foundation [of ideas], to say `Let's make the foundation as strong as we can make it.'\hspace{0.01in}''

The GPL's expanding popularity and continuing gravitational strength are the best tributes to the foundation laid by Stallman and his GNU colleagues. While Stallman was never the sole person in the world releasing free software, he nevertheless can take sole credit for building the free software movement's ethical framework. Whether or not other modern programmers feel comfortable working inside that framework is immaterial. The fact that they even have a choice at all is Stallman's greatest legacy.

Discussing Stallman's legacy at this point seems a bit premature. Stallman, 48 at the time of this writing, still has a few years left to add to or subtract from that legacy. Still, the momentum of the free software movement makes it tempting to examine Stallman's life outside the day-to-day battles of the software industry and within a more august, historical setting.

To his credit, Stallman refuses all opportunities to speculate about this. ``I've never been able to work out detailed plans of what the future was going to be like,'' says Stallman, offering his own premature epitaph. ``I just said `I'm going to fight. Who knows where I'll get?'\hspace{0.01in}''

There's no question that in picking his fights, Stallman has alienated the very people who might otherwise have been his greatest champions, had he been willing to fight for their views instead of his own. It is also a testament to his forthright, ethical nature that many of Stallman's erstwhile political opponents still manage to put in a few good words for him when pressed. The tension between Stallman the ideologue and Stallman the hacker genius, however, leads a biographer to wonder: how will people view Stallman when Stallman's own personality is no longer there to get in the way?

In early drafts of this book, I dubbed this question the ``100 year'' question. Hoping to stimulate an objective view of Stallman and his work, I asked various software-industry luminaries to take themselves out of the current timeframe and put themselves in a position of a historian looking back on the free software movement 100 years in the future. From the current vantage point, it is easy to see similarities between Stallman and past Americans who, while somewhat marginal during their lifetime, have attained heightened historical importance in relation to their age. Easy comparisons include Henry David Thoreau, transcendentalist philosopher and author of \textit{Civil Disobedience}, and John Muir, founder of the Sierra Club and progenitor of the modern environmental movement. It is also easy to see similarities in men like William Jennings Bryan, a.k.a. ``The Great Commoner,'' leader of the populist movement, enemy of monopolies, and a man who, though powerful, seems to have faded into historical insignificance.

Although not the first person to view software as public property, Stallman is guaranteed a footnote in future history books thanks to the GPL. Given that fact, it seems worthwhile to step back and examine Richard Stallman's legacy outside the current time frame. Will the GPL still be something software programmers use in the year 2102, or will it have long since fallen by the wayside? Will the term ``free software'' seem as politically quaint as ``free silver'' does today, or will it seem eerily prescient in light of later political events?

Predicting the future is risky sport. Stallman refuses, saying that asking what people will think in 100 years presumes we have no influence over it.  The question he prefers is, ``What should we do to make a better future?''  But most people, when presented with the predictive question, seemed eager to bite.
 
``One hundred years from now, Richard and a couple of other people are going to deserve more than a footnote,'' says Moglen. ``They're going to be viewed as the main line of the story.''

The ``couple of other people'' Moglen nominates for future textbook chapters include John Gilmore, who beyond contributing in various ways to free software has founded the Electronic Frontier Foundation, and Theodor Holm Nelson, a.k.a. Ted Nelson, author of the 1982 book, \textit{Literary Machines}. Moglen says Stallman, Nelson, and Gilmore each stand out in historically significant, nonoverlapping ways. He credits Nelson, commonly considered to have coined the term ``hypertext,'' for identifying the predicament of information ownership in the digital age. Gilmore and Stallman, meanwhile, earn notable credit for identifying the negative political effects of information control and building organizations -- the Electronic Frontier Foundation in the case of Gilmore and the Free Software Foundation in the case of Stallman -- to counteract those effects. Of the two, however, Moglen sees Stallman's activities as more personal and less political in nature.

``Richard was unique in that the ethical implications of unfree software were particularly clear to him at an early moment,'' says Moglen. ``This has a lot to do with Richard's personality, which lots of people will, when writing about him, try to depict as epiphenomenal or even a drawback in Richard Stallman's own life work.''

Gilmore, who describes his inclusion between the erratic Nelson and the irascible Stallman as something of a ``mixed honor,'' nevertheless seconds the Moglen argument. Writes Gilmore:

\begin{quote}
My guess is that Stallman's writings will stand up as well as Thomas Jefferson's have; he's a pretty clear writer and also clear on his principles\ldots Whether Richard will be as influential as Jefferson will depend on whether the abstractions we call ``civil rights'' end up more important a hundred years from now than the abstractions that we call ``software'' or ``technically imposed restrictions.''
\end{quote}

Another element of the Stallman legacy not to be overlooked, Gilmore writes, is the collaborative software-development model pioneered by the GNU Project. Although flawed at times, the model has nevertheless evolved into a standard within the software-development industry. All told, Gilmore says, this collaborative software-development model may end up being even more influential than the GNU Project, the GPL License, or any particular software program developed by Stallman:

\begin{quote}
Before the Internet, it was quite hard to collaborate over distance on software, even among teams that know and trust each other. Richard pioneered collaborative development of software, particularly by disorganized volunteers who seldom meet each other. Richard didn't build any of the basic tools for doing this (the TCP protocol, email lists, diff and patch, tar files, RCS or CVS or remote-CVS), but he used the ones that were available to form social groups of programmers who could effectively collaborate.
\end{quote}

Stallman thinks that evaluation, though positive, misses the point. ``It emphasizes development methods over freedom, which reflects the values of open source rather than free software.  If that is how future users look back on the GNU Project, I fear it will lead to a world in which developers maintain users in bondage, and let them aid development occasionally as a reward, but never take the chains off them.''

Lawrence Lessig, Stanford law professor and author of the 2001 book, \textit{The Future of Ideas}, is similarly bullish. Like many legal scholars, Lessig sees the GPL as a major bulwark of the current so-called ``digital commons,'' the vast agglomeration of community-owned software programs, network and telecommunication standards that have triggered the Internet's exponential growth over the last three decades. Rather than connect Stallman with other Internet pioneers, men such as Vannevar Bush, Vinton Cerf, and J. C. R. Licklider who convinced others to see computer technology on a wider scale, Lessig sees Stallman's impact as more personal, introspective, and, ultimately, unique:

\begin{quote}
[Stallman] changed the debate from ``is'' to ``ought.'' He made people see how much was at stake, and he built a device to carry these ideals forward\ldots That said, I don't quite know how to place him in the context of Cerf or Licklider. The innovation is different. It is not just about a certain kind of code, or enabling the Internet. [It's] much more about getting people to see the value in a certain kind of Internet. I don't think there is anyone else in that class, before or after.
\end{quote}

Not everybody sees the Stallman legacy as set in stone, of course. Eric Raymond, the open source proponent who feels that Stallman's leadership role has diminished significantly since 1996, sees mixed signals when looking into the 2102 crystal ball:

\begin{quote}
I think Stallman's artifacts (GPL, Emacs, GCC) will be seen as revolutionary works, as foundation-stones of the information world. I think history will be less kind to some of the theories from which RMS operated, and not kind at all to his personal tendency towards territorial, cult-leader behavior.
\end{quote}

As for Stallman himself, he, too, sees mixed signals:

\begin{quote}
What history says about the GNU Project, twenty years from now, will depend on who wins the battle of freedom to use public knowledge. If we lose, we will be just a footnote. If we win, it is uncertain whether people will know the role of the GNU operating system -- if they think the system is ``Linux,'' they will build a false picture of what happened and why.

But even if we win, what history people learn a hundred years from now is likely to depend on who dominates politically.
\end{quote}

Searching for his own 19th-century historical analogy, Stallman summons the figure of John Brown, the militant abolitionist regarded as a hero on one side of the Mason Dixon line and a madman on the other.

John Brown's slave revolt never got going, but during his subsequent trial he effectively roused national demand for abolition. During the Civil War, John Brown was a hero; 100 years after, and for much of the 1900s, history textbooks taught that he was crazy. During the era of legal segregation, while bigotry was shameless, the U.S. partly accepted the story that the South wanted to tell about itself, and history textbooks said many untrue things about the Civil War and related events.

Such comparisons document both the self-perceived peripheral nature of Stallman's current work and the binary nature of his current reputation. It's hard to see Stallman's reputation falling to the same level of infamy as Brown's did during the post-Reconstruction period. Stallman, despite his occasional war-like analogies, has done little to inspire violence. Still, it is easy to envision a future in which Stallman's ideas wind up on the ash-heap.\endnote{RMS: Sam Williams' further words here, ``In fashioning the free software cause not  not as a mass movement but as a collection of private battles against the forces of proprietary temptation,'' do not fit the facts. Ever since the first announcement of the GNU Project, I have asked the public to support the cause. The free software movement aims to be a mass movement, and the only question is whether it has enough supporters to qualify as ``mass.''  As of 2009, the Free Software Foundation has some 3000 members that pay the hefty dues, and over 20,000 subscribers to its monthly e-mail newsletter.}

Then again, it is that very will that may someday prove to be Stallman's greatest lasting legacy. Moglen, a close observer over the last decade, warns those who mistake the Stallman personality as counter-productive or epiphenomenal to the ``artifacts'' of Stallman's life. Without that personality, Moglen says, there would be precious few artifacts to discuss. Says Moglen, a former Supreme Court clerk:

\begin{quote}
Look, the greatest man I ever worked for was Thurgood Marshall. I knew what made him a great man. I knew why he had been able to change the world in his possible way. I would be going out on a limb a little bit if I were to make a comparison, because they could not be more different: Thurgood Marshall was a man in society, representing an outcast society to the society that enclosed it, but still a man in society. His skill was social skills. But he was all of a piece, too. Different as they were in every other respect, the person I now most compare him to in that sense -- of a piece, compact, made of the substance that makes stars, all the way through -- is Stallman.
\end{quote}

In an effort to drive that image home, Moglen reflects on a shared moment in the spring of 2000. The success of the VA Linux IPO was still resonating in the business media, and a half dozen issues related to free software were swimming through the news. Surrounded by a swirling hurricane of issues and stories each begging for comment, Moglen recalls sitting down for lunch with Stallman and feeling like a castaway dropped into the eye of the storm. For the next hour, he says, the conversation calmly revolved around a single topic: strengthening the GPL.

``We were sitting there talking about what we were going to do about some problems in Eastern Europe and what we were going to do when the problem of the ownership of content began to threaten free software,'' Moglen recalls. ``As we were talking, I briefly thought about how we must have looked to people passing by. Here we are, these two little bearded anarchists, plotting and planning the next steps. And, of course, Richard is plucking the knots from his hair and dropping them in the soup and behaving in his usual way. Anybody listening in on our conversation would have thought we were crazy, but I knew: I knew the revolution's right here at this table. This is what's making it happen. And this man is the person making it happen.''

Moglen says that moment, more than any other, drove home the elemental simplicity of the Stallman style.

``It was funny,'' recalls Moglen. ``I said to him, `Richard, you know, you and I are the two guys who didn't make any money out of this revolution.' And then I paid for the lunch, because I knew he didn't have the money to pay for it.''\endnote{RMS: I never refuse to let people treat me to a meal, since my pride is not based on picking up the check.  But I surely did have the money to pay for lunch.  My income, which comes from around half the speeches I give, is much less than a law professor's salary, but I'm not poor.}

\theendnotes
\setcounter{endnote}{0}
