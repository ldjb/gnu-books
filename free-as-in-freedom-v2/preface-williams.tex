%% Copyright (c) 2002, 2010 Sam Williams
%% Copyright (c) 2010 Richard M. Stallman
%% Permission is granted to copy, distribute and/or modify this
%% document under the terms of the GNU Free Documentation License,
%% Version 1.3 or any later version published by the Free Software
%% Foundation; with no Invariant Sections, no Front-Cover Texts, and
%% no Back-Cover Texts. A copy of the license is included in the
%% file called ``gfdl.tex''.


\chapter{Preface by Sam Williams}

This summer marks the 10th anniversary of the email exchange that set
in motion the writing of \textit{Free as in Freedom: Richard
  Stallman's Crusade for Free Software} and, by extension, the work
prefaced here, \textit{Richard Stallman and the Free Software
  Revolution}.

Needless to say, a lot has changed over the intervening decade.

Originally conceived in an era of American triumphalism, the book's
main storyline -- about one man's Jeremiah-like efforts to enlighten
fellow software developers as to the ethical, if not economic,
shortsightedness of a commercial system bent on turning the free range
intellectual culture that gave birth to computer science into a rude
agglomeration of proprietary gated communities -- seems almost
nostalgic, a return to the days when the techno-capitalist system
seemed to be working just fine, barring the criticism of a few
outlying skeptics.

Now that doubting the system has become almost a common virtue, it  
helps to look at what narrative threads, if any, remained consistent  
over the last ten years.

While I don't follow the software industry as closely as I once did,  
one thing that leaps out now, even more than it did then, is the ease  
with which ordinary consumers have proven willing to cede vast swaths  
of private information and personal user liberty in exchange for  
riding atop the coolest technology ``platform'' or the latest networking  
trend.

A few years ago, I might have dubbed this the ``iPod Effect,'' a
shorthand salute to Apple co-founder Steve Jobs' unrivaled success in
getting both the music industry and digital music listeners to put
aside years of doubt and mutual animosity to rally around a single,
sexy device -- the Apple iPod -- and its restrictive licensing regime,
iTunes. Were I pitching the story to a magazine or newspaper nowadays,
I d probably have to call it the ``iPad Effect'' or maybe the ``Kindle
Effect'' both in an attempt to keep up with the evolving brand names
and to acknowledge parallel, tectonic shifts in the realm of daily
journalism and electronic book publishing.

Lest I appear to be gratuitously plugging the above-mentioned brand
names, RMS suggests that I offer equal time to a pair of web sites that
can spell out their many disadvantages, especially in the realm of
software liberty. I have agreed to this suggestion in the spirit of
equal time. The web sites he recommends are \url{DefectiveByDesign.org}
and \url{BadVista.org}.

Regardless of title, the notion of corporate brand as sole guarantor  
of software quality in a swiftly changing world remains a hard one to  
dislodge, even at a time when most corporate brands are trading at or  
near historic lows.

Ten years ago, it wasn't hard to find yourself at a technology
conference listening in on a conversation (or subjected to direct
tutelage) in which some old-timer, Richard Stallman included, offered
a compelling vision of an alternate possibility. It was the job of
these old-timers, I ultimately realized, to make sure we newbies in
the journalism game recognized that the tools we prided ourselves in
finally knowing how to use -- Microsoft Word, PowerPoint, Internet
Explorer, just to name a few popular offerings from a single oft-cited
vendor -- were but a pale shadow of towering edifice the original
architects of the personal computer set out to build.

Nowadays, it's almost as if the opposite situation is at hand. The
edifice is now a sprawling ecosystem, a jungle teeming with ideas but
offering only a few stable niches for sustainable growth. While one
can still find plenty of hackers willing to grumble about, say,
Vista's ongoing structural flaws, Apple's dictatorial oversight of the
iPhone App Store or Google's shifting definition of the word ``evil''
-- each year brings with it a fresh crop of ``digital native''
consumers willing to trust corporate guidance in this Hobbesian realm.
Maybe that's because many of the problems that once made using your
desktop computer such a teeth-grinding experience have largely been
paved over with the help of free software.

Whatever. As consumer software reliability has improved, the race to
stay one step ahead of consumer taste has put application developers
in an even tighter embrace with moneyed interests. I'm not saying that
the hacker ethos no longer exists or that it has even weakened in any
noticeable way. I'm just saying that I doubt the programmer who
generated the Facebook algorithm that rewrites the ``info'' pages so
that each keyword points to a sponsored page, with an 80-percent
semantic error rate to boot, spends much time in his new Porsche
grousing about what the program really could have achieved if only the
``suits'' hadn't gotten in the way.

True, millions of people now run mostly free software on their
computers with many running free software exclusively. From an
ordinary consumer perspective, however, terms like ``software'' and
``computer'' have become increasingly distant. Many 2010-era cell
phones could give a 2000-era laptop a run for its money in the
functionality department. And yet, when it comes time to make a cell
phone purchase, how many users lend any thought to the computer or
software operating system making that functionality possible? The vast
majority of modern phone users base their purchasing decisions almost
entirely on the number of applications offered, the robustness of the
network and, most important of all, the monthly service plan.

Getting a consumer in this situation to view his or her software  
purchase through the lens of personal liberty, as opposed to personal  
convenience, is becoming, if not more difficult, certainly a more  
complex endeavor.

Given this form of pessimistic introduction, why should anyone want go  
on and read this book?

I can offer two major reasons.

The first reason is a personal one. As noted in the Epilogue of
\textit{Free as in Freedom}, Richard and I parted on less than cordial
terms shortly before the publication of that book. The fault, in large
part, was mine. Having worked with Richard to make sure that my
biographical sketch didn't run afoul of free software principles -- an
effort that, I'm proud to say, made \textit{Free as in Freedom} one of
the first works to employ the GNU Free Documentation License (GFDL) as
a copyright mechanism -- I abruptly ended the cooperative relationship
when it came time to edit the work and incorporate Richard's lengthy
list of error corrections and requests for clarification.

Though able to duck behind my own principles of authorial independence
and journalistic objectivity, I have since come to lament not begging
the book's publisher -- O'Reilly and Associates -- for additional time.
Because O'Reilly had already granted my one major stipulation -- the
GFDL -- and had already put up with a heavy stream of last-minute
changes on my part, however, I was hesitant to push my luck.

In the years immediately following the publication of \textit{Free as
  in Freedom}, I was able to justify my decision by noting that the
GFDL, just like the GNU General Public License in the software realm,
makes it possible for any reader to modify the book and resell it as a
competitive work. As Ernest Hemingway once put it, ``the first draft
of anything is shit.'' If Stallman or others within the hacker
community saw \textit{Free as in Freedom} as a first draft at best,
well, at least I had spared them the time and labor of generating
their own first draft.

Now that Richard has indeed delivered what amounts to a significant
rewrite, I can only but remain true to my younger self and endorse the
effort. Indeed, I salute it. My only remaining hope is that, seeing as
how Richard's work doesn't show any sign of slowing, additional
documentation gets added to the mix.

Before moving on to the next reason, I should note that one of the pleasant
by-products of this book is a re-opening of email communication
channels between Richard and myself. The resulting communication has
reacquainted me with the razor-sharp Stallman writing style.

An illustrative and perhaps amusing anecdote for anyone out there who
has wrangled with Richard in text: In the course of discussing the
passage in which I observe and document the process of Richard losing
his cool amid the rush hour traffic of Kihei, Maui, a passage that
served as the basis for Chapter 7 (``A Brief Journey through Hacker
Hell'') in the original book, I acknowledged a common complaint among
the book's reviewers -- namely, that the episode seemed out of place,
a fragment of magazine-style profile interrupting a book-length
biography. I told Richard that he could discard the episode for that
reason alone but noted that my decision to include it was based on two
justifications. First, it offered a glimpse of the Stallman temper,
something I'd been warned about but had yet to experience in a
firsthand manner. Second, I felt the overall scene possessed a certain
metaphorical value. Hence the chapter title.

Stallman, to my surprise, agreed on both counts. His concern lay more
in the two off-key words. At one point I quote him accusing the lead
driver of our two-vehicle caravan with ``deliberately'' leading us
down a dead-end street, an accusation that, if true, suggested a level
of malice outside the bounds of the actual situation. Without the
benefit of a recorded transcript -- I only had a notebook at the time,
I allowed that it was likely I'd mishandled Stallman's actual wording
and had made it more hurtful than originally intended.

On a separate issue, meanwhile, Stallman questioned his quoted use of
the word ``fucking.'' Again, I didn't have the moment on tape, but I
wrote back that I distinctly recalled an impressive display of
profanity, a reminder of Richard's New York roots, and was willing to
stand by that memory.

An email response from Richard, received the next day, restated the
critique in a way that forced me to go back and re-read the first
message. As it turned out, Stallman wasn't so much objecting to the
``fuck'' as the ``-ing'' portion of the quote.

``Part of the reason I doubt [the words] is that they involve using
fucking as an adverb,'' Stallman wrote. ``I have never spoken that
way. So I am sure the words are somewhat altered.''

Touché.

The second reason a person should feel compelled to read this book
cycles back to the opening theme of this preface -- how different a
future we face in 2010 compared to the one we were still squinting our
eyes to see back in 2000. I ll be honest: Like many Americans (and
non-Americans), my worldview was altered by the events of September
11, 2001, so much so that it wasn't much longer after the publication
of \textit{Free as in Freedom} that my attention drifted sharply away
from the free software movement and Stallman's efforts to keep it on
course. While I have managed to follow the broad trends and major
issues, the day-to-day drama surrounding software standards, software
copyrights and software patents has become something I largely skip
over -- the Internet news equivalent of the Water Board notes in the
local daily newspaper, in other words.

[RMS: The September 2001 attacks, not mentioned later in the book,
  deserve brief comment here. Far from ``changing everything,'' as
  many proclaim, the attacks have, in fact, changed very little in the
  U.S.: There are still scoundrels in power who hate our freedoms. The
  only major difference is that they can now cite ``terrorists'' as an
  excuse for laws to take them away. See the political notes on
  \url{stallman.org} for more about this.]

This is a lamentable development in large part because, ten years in,
I finally see the maturing 21st century in what I believe to be a
clear light. Again, if this were a pitch letter to some editor, I'd
call it ``The Process Century.''

By that I mean I we stand at a rare point in history where, all
cynicism aside, the power to change the world really does delegate
down to the ordinary citizen's level. The catch, of course, is that
the same power that belongs to you also belongs to everyone else.
Where in past eras one might have secured change simply by winning the
sympathies of a few well-placed insiders, today's reformer must bring
into alignment an entire vector field of competitive ideas and
interests. In short, being an effective reformer nowadays requires
more than just titanic stamina and a willingness to cry out in the
wilderness for a decade or more, it requires knowing how to articulate
durable, scalable ideas, how to beat the system at its own game.

On all counts, I would argue that Richard M. Stallman, while maybe not  
the archetype, is at the very least an ur-type of the successful  
reformer just described.

While some might lament a future in which every problem seems to take  
a few decades of committee meetings and sub-committee hearings just to  
reach the correction stage, I, for one, see the alternative  -- a  
future so responsive to individual or small group action that some  
self-appointed actor finally decides to put that responsiveness to the  
test -- as too chilling to contemplate.

In short, if you are the type of person who, like me, hopes to see the
21st century follow a less bloody course than the 20th century, the
Water Board -- in its many frustrating guises -- is where that battle
is currently being fought. As hinted by the Virgil-inspired epigraph
introducing the book's first chapter, I've always held out hope that
this book might in some way become a sort of epic poem for the
Internet Age. Built around a heroic but flawed central figure, its
authorial stamp should be allowed to blur with age.

On that note, I would like to end this preface the same way I always
end this preface -- with a request for changes and contributions from
any reader wishing to improve the text. \nameref{Appendix B} offers a
guide on your rights as a reader to submit changes, make corrections,
or even create your own spin-off version of the book. If you prefer to
simply run the changes through Richard or myself, you can find the
pertinent contact information on the Free Software Foundation web site.

In the meantime, good luck and enjoy the book!

\vspace{0.5in}
\noindent Sam Williams\\
\noindent Staten Island, USA
